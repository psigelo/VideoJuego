\documentclass[11pt]{book}
\usepackage[utf8]{inputenc}
\usepackage{mathtools} % Muy util
\usepackage{amsfonts} % Para usar fuentes de texto matemáticas
\usepackage{mathrsfs} % para usar fuentes de texto matemáticas (esa para clase dos veces derivable por ejemplo)
\usepackage{float} % Para que las imágenes queden en su lugar
\usepackage{color} % Para colorear text
\usepackage{microtype} % Para corregir errores de justificación (o sea para que se alinee sólo el texto de forma optima)
\usepackage{enumerate} % Para enumerar con letras
\usepackage[top=2in, bottom=1.5in, left=1in, right=1in]{geometry} % así se definen los márgenes
\usepackage{xcolor}
\usepackage{sectsty} % Para ponerle colores a las secciones.
\usepackage{makeidx} % Para la creación de índices.

\chapterfont{\color{blue}}  % sets colour of chapters
\sectionfont{\color{cyan}}  % sets colour of sections

\makeindex

\begin{document}
\title{Documentación Juego}
\author{Pascal Sigel}
\maketitle
\printindex
\chapter{Introducción.}

El objetivo de este proyecto es realizar una buena estructura en C++ de lo que debe ser un video juego, buscando modularidad en lo que es procesamiento y lo que es gráfico para que pueda fácilmente implementarse en otras arquitecturas.

\chapter{Modelo.}









\end{document}